\documentclass{article}
\usepackage{tikz}
\usetikzlibrary{shapes,shadows,arrows}

\usepackage{fp}

\begin{document}
\clearpage
\thispagestyle{empty}

\pgfdeclarelayer{background}
\pgfdeclarelayer{main}
\pgfdeclarelayer{foreground}
\pgfsetlayers{background,main,foreground}
\definecolor{pear}{RGB}{255,69,00}

% % % % % % % % [Begin] These are configurable % % % % % % %
\FPeval{\lw}{20}        % linewidth
\FPeval{\rone}{4}    % r1: radius of circle c1
\FPeval{\rtwo}{3.14}    % r2: 3.14 pi
\FPeval{\rthree}{18.47} % r3. 18.47 = 32 * Euler's Constant
\FPeval{\rfour}{2.349461690234380417} % to be fixed for higher precision
\FPeval{\a}{18}         % alpha
\FPeval{\b}{36}         % beta
\FPeval{\c}{18}         % gamma
\FPeval{\ydelta}{1.2}   % distance from the top of the pear body to the bottom of the bar
\FPeval{\ll}{1.602}     % lengh of the x-leg of the bar
% % % % % % % % [End] These are configurable % % % % % % %

% % % % % % % % [Begin] Obsolete % % % % % % %
\FPeval{\eps}{0}        % 0.016. obsolete
\FPeval{\cw}{0.345}     % circle radius equals to 1/2 linewidth. obsolete
% % % % % % % % [End] Obsolete % % % % % % %

\FPeval{\xone}{(\rone)*sin(\a/180*pi)}
\FPeval{\yone}{(\rone)*(cos(\a/180*pi)-1)}

%\FPeval{\c2x}{\x1 + \r2*(sin(\a/180*pi))}
%\FPeval{\c2y}{\y1 + \r2*(cos(\a/180*pi))}
\FPeval{\xtwo}{(\xone) + (\rtwo)*(sin(\a/180*pi)+cos(\b/180*pi))}
\FPeval{\ytwo}{(\yone) + (\rtwo)*(cos(\a/180*pi)+sin(\b/180*pi))}

\FPeval{\xthree}{(\xtwo) + (\rthree)*((cos(\b/180*pi)-cos(\c/180*pi)))}
\FPeval{\ythree}{(\ytwo) + (\rthree)*((sin(\b/180*pi)-sin(\c/180*pi)))}

\FPeval{\rfour}{(\xthree)/cos(\c/180*pi)} % fix r4
%\FPeval{\xfour}{0}
\FPeval{\xright}{(\xone) + (\rtwo)*(sin(\a/180*pi)+1)}
\FPeval{\xbottom}{0}
\FPeval{\ybottom}{(\yone) + (\rtwo)*(cos(\a/180*pi)-1)}
\FPeval{\xtop}{0}
\FPeval{\ytopl}{(\ythree) + (\rfour)*sin(pi/4-\c/180*pi/2)}
\FPeval{\ytop}{(\ytopl) + (\ydelta) + (\ll)}

\FPeval{\xseed}{0.916}
\FPeval{\yseed}{(\ybottom) + (\ytopl-\ybottom)*(1-0.618)}

\begin{tikzpicture}
\begin{pgfonlayer} {foreground}
%\node [elli, fill=white] (face) {};

%\draw [line width=\lw]  (\eps,0) arc (90:90+\a:\rone); % 2*e

\draw [line join=round,line cap=round,thick,line width=\lw]  (0,0) arc (90:90-\a:\rone) -- % can also be (-eps,0)
(\xone,\yone) arc (-90-\a:\b:\rtwo) --
(\xtwo,\ytwo) arc (180+\b:180+\c:\rthree) --
(\xthree,\ythree) arc (\c:180-\c:\rfour) --
(-\xthree,\ythree) arc (-\c:-\b:\rthree) --
(-\xtwo,\ytwo) arc (180-\b:270+\a:\rtwo) --
(-\xone,\yone) arc (90+\a:90:\rone);

\fill [fill=black,shift={(\xseed,\yseed)},rotate=\b,scale=0.1]  (0,0) arc (90:90-\a:\rone) -- % can also be (-eps,0)
(\xone,\yone) arc (-90-\a:\b:\rtwo) --
(\xtwo,\ytwo) arc (180+\b:180+\c:\rthree) --
(\xthree,\ythree) arc (\c:180-\c:\rfour) --
(-\xthree,\ythree) arc (-\c:-\b:\rthree) --
(-\xtwo,\ytwo) arc (180-\b:270+\a:\rtwo) --
(-\xone,\yone) arc (90+\a:90:\rone);

\fill [fill=black,shift={(-\xseed,\yseed)},rotate=-\b,scale=0.1]  (0,0) arc (90:90-\a:\rone) -- % can also be (-eps,0)
(\xone,\yone) arc (-90-\a:\b:\rtwo) --
(\xtwo,\ytwo) arc (180+\b:180+\c:\rthree) --
(\xthree,\ythree) arc (\c:180-\c:\rfour) --
(-\xthree,\ythree) arc (-\c:-\b:\rthree) --
(-\xtwo,\ytwo) arc (180-\b:270+\a:\rtwo) --
(-\xone,\yone) arc (90+\a:90:\rone);


\draw[line join=round,line cap=round,thick,line width=\lw] (0,\ytopl+\ydelta) to (\ll,\ytop); 
%\draw [line width=\lw] (0,12.430) to (1.602,14.032); 					% 1.602: elementary charge

%\draw [fill] (0,12.430) circle [radius=\cw];  % 4*Brun's Constant
%\draw [fill] (1.602,14.032) circle [radius=\cw];

%\draw [fill] (0.916,4.1314) circle [radius=\cw]; % 0.916: Catalan Constant
%\draw [fill] (-0.916,4.1314) circle [radius=\cw];
%\draw [line width=\lw] (0.916,4) arc (-90:-45:2.346);
%\draw [line width=\lw] (-0.916,4) arc (-90:-135:2.346);

%\draw [fill] (0.571,5.1932) -- (0.571,4.1314) arc (-90:-54:1.0618) -- cycle;
%\draw [fill] (-0.571,5.1932) -- (-0.571,4.1314) arc (-90:-126:1.0618) -- cycle;

% Let (x0, y0) denote the center of the right "seed" circle, (x1, y1) be the center of the circular sector, (x1, x2) denote the left intersection point of the two.
% Let r be the radius of the seed circle. 
% Let alpha be the angel between (x1, y1) -- (x0, y0) and (x1, y1) -- (x0, y0). 
% Let beta be the angel between the plumbline through (x1, y1) and (x1, y1) -- (x0, y0). 
% For example, alpha = 18, beta = 18 --- this is the Golden Triangle case. 
% sin(alpha) = r / sqrt((x1-x0)^2+(y1-y0)^2); tan(beta) = (y1-y0) / (x0-x1). Solve this, we get: 
% x1 = x0 - r*cos(beta)/sin(alpha)		0.916 - 0.345*sin(18)/sin(18)	   0.571...
% y1 = y0 + (x0-x1)/tan(beta)		4.1314 + (0.916-0.571)/tan(18)		5.1932...
% Since: (x2-x0)^2 + (y2-y0)^2 = r^2; (x2-x1)/(y1-y2) = tan(beta-alpha)		(x-0.916)^2 + (y-4)^2 = 0.345^2; (x-0.571)/(5.0618-y) = tan(18)
% x2 = ;  y2 = . {x == 1.02261, y == 3.67189}

% Bounding box
%\draw [line width=1] (-\xright,\ybottom) to (\xright,\ybottom);
%\draw [line width=1] (-\xright,\ytop) to (\xright,\ytop);
%\draw [line width=1] (-\xright,\ytopl) to (\xright,\ytopl);
%\draw [line width=1] (-\xright,\ybottom) to (-\xright,\ytop);
%\draw [line width=1] (\xright,\ybottom) to (\xright,\ytop);
% Bounding box

% -0.350 + (11.3845+0.350)*(1-0.618) = 4.1314
% -0.350 + (14.032+0.350)*(1-0.618) = 5.1439

%\node [c1, xshift=0, yshift=0]{};

%\node [c1, xshift=-55.80, yshift=-9.20]{};
%\node [c2, xshift=55.80, yshift=-14.44, rotate=81, circular sector angle=18, line width=1.0mm]{};


%\node [c1, xshift=-55.80, yshift=89.20]{};


%\draw [fill=black] (0,20) -- (3mm,0mm) arc (0:30:3mm) -- (0,0);

%torso
%\draw [line width=5mm](face.230) to[out=260, in=150] +(0.75in,-3.15in);
%\draw [line width=5mm](face.310) to[out=280, in=30] +(-0.75in,-3.15in);
%eyes
%\node [vrutt, xshift=-5em, yshift=9mm] (lefteye) {};
%\node [vrutt, xshift=5em, yshift=9mm] (righteye) {};
% Smile
%\draw [line width=5mm] (-2.0,-1.0) to[out=320, in=220] (2.0,-1.0);
%Antenna
%\draw[line width=5mm](-0.5,3.76) -- +(1cm, 2.5cm) -- +(3.5cm, 2cm);
%\node [vrutt, fill=none, draw=black, above of=face, yshift=4.65cm, xshift=3.5cm, minimum height=0.5in] (antenna){};
%Text
%\node (face.275)[yshift=-3in] (text){\Huge \textbf{\LaTeX}};

% added by alan
%\draw (8,0) arc [start angle=0, end angle=270,
%x radius=1cm, y radius=5mm] -- cycle;

\end{pgfonlayer}

\begin{pgfonlayer} {background}
%Ears
%\draw [line width=4mm] (4.4,1.3) arc (-80:315:1);
%\draw [line width=4mm] (-4.3,1.3) arc (-80:315:1);
%hands
%\draw [line width=4mm] (3.05,-7.8) arc (-70:80:2.3);
%\draw [line width=4mm] (-3.05,-7.8) arc (250:90:2.3);

\end{pgfonlayer}
\end{tikzpicture}

\end{document}
