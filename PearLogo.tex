\documentclass{article}
\usepackage{tikz}
\usetikzlibrary{shapes,shadows,arrows}

\begin{document}
\clearpage
\thispagestyle{empty}

\pgfdeclarelayer{background}
\pgfdeclarelayer{main}
\pgfdeclarelayer{foreground}
\pgfsetlayers{background,main,foreground}
\definecolor{orangered}{RGB}{255,69,00}
\tikzstyle{vrutt}=[draw=orangered, fill=orangered, circle,minimum height=0.5in, line width=5mm]
\tikzstyle{elli}=[draw, ellipse, minimum height=2.85in, text width=2.95in, text centered, line width=5mm]
\tikzstyle{c1}=[draw, circle, fill=black, minimum height=6.88mm]
\tikzstyle{c2}=[draw, circular sector, fill=black]

\begin{tikzpicture}
\begin{pgfonlayer} {foreground}
%\node [elli, fill=white] (face) {};

\draw [line width=20] (0.016,0) arc (90:108:4.36); % 2*e
\draw [line width=20] (-0.016,0) arc (90:72:4.36);

\draw [line width=20] (-1.324,-0.211) arc (288:144:3.14); % 3.14
\draw [line width=20] (1.324,-0.211) arc (-108:36:3.14);

\draw [line width=20] (-4.837,4.617) arc (-36:-18:18.47); % 32 * Euler's Constant
\draw [line width=20] (4.837,4.617) arc (216:198:18.47);

\draw [line width=20] (0.016,11.3815) arc (90:162:2.346);
\draw [line width=20] (-0.016,11.3815) arc (90:18:2.346);

\draw [line width=20] (0,12.430) to (1.602,14.032); % 1.602: elementary charge

\draw [fill] (0,12.430) circle [radius=0.345];  % 4*Brun's Constant
\draw [fill] (1.602,14.032) circle [radius=0.345];

\draw [fill] (0.916,4.1314) circle [radius=0.345]; % 0.916: Catalan Constant
\draw [fill] (-0.916,4.1314) circle [radius=0.345];
%\draw [line width=20] (0.916,4) arc (-90:-45:2.346);
%\draw [line width=20] (-0.916,4) arc (-90:-135:2.346);

\draw [fill] (0.571,5.1932) -- (0.571,4.1314) arc (-90:-54:1.0618) -- cycle;
\draw [fill] (-0.571,5.1932) -- (-0.571,4.1314) arc (-90:-126:1.0618) -- cycle;

% Let (x0, y0) denote the center of the right "seed" circle, (x1, y1) be the center of the circular sector, (x1, x2) denote the left intersection point of the two.
% Let r be the radius of the seed circle. 
% Let alpha be the angel between (x1, y1) -- (x0, y0) and (x1, y1) -- (x0, y0). 
% Let beta be the angel between the plumbline through (x1, y1) and (x1, y1) -- (x0, y0). 
% For example, alpha = 18, beta = 18 --- this is the Golden Triangle case. 
% sin(alpha) = r / sqrt((x1-x0)^2+(y1-y0)^2); tan(beta) = (y1-y0) / (x0-x1). Solve this, we get: 
% x1 = x0 - r*cos(beta)/sin(alpha)		0.916 - 0.345*sin(18)/sin(18)	   0.571...
% y1 = y0 + (x0-x1)/tan(beta)		4.1314 + (0.916-0.571)/tan(18)		5.1932...
% Since: (x2-x0)^2 + (y2-y0)^2 = r^2; (x2-x1)/(y1-y2) = tan(beta-alpha)		(x-0.916)^2 + (y-4)^2 = 0.345^2; (x-0.571)/(5.0618-y) = tan(18)
% x2 = ;  y2 = . {x == 1.02261, y == 3.67189}


\draw [line width=1] (-6,11.3815) to (6,11.3815);
\draw [line width=1] (-6,14.032) to (6,14.032);
\draw [line width=1] (-6,-0.350) to (6,-0.350);
\draw [line width=1] (-5.4,-1) to (-5.4,16);
\draw [line width=1] (5.4,-1) to (5.4,16);

% -0.350 + (11.3845+0.350)*(1-0.618) = 4.1314
% -0.350 + (14.032+0.350)*(1-0.618) = 5.1439

%\node [c1, xshift=0, yshift=0]{};

%\node [c1, xshift=-55.80, yshift=-9.20]{};
%\node [c2, xshift=55.80, yshift=-14.44, rotate=81, circular sector angle=18, line width=1.0mm]{};


%\node [c1, xshift=-55.80, yshift=89.20]{};


%\draw [fill=black] (0,20) -- (3mm,0mm) arc (0:30:3mm) -- (0,0);

%torso
%\draw [line width=5mm](face.230) to[out=260, in=150] +(0.75in,-3.15in);
%\draw [line width=5mm](face.310) to[out=280, in=30] +(-0.75in,-3.15in);
%eyes
%\node [vrutt, xshift=-5em, yshift=9mm] (lefteye) {};
%\node [vrutt, xshift=5em, yshift=9mm] (righteye) {};
% Smile
%\draw [line width=5mm] (-2.0,-1.0) to[out=320, in=220] (2.0,-1.0);
%Antenna
%\draw[line width=5mm](-0.5,3.76) -- +(1cm, 2.5cm) -- +(3.5cm, 2cm);
%\node [vrutt, fill=none, draw=black, above of=face, yshift=4.65cm, xshift=3.5cm, minimum height=0.5in] (antenna){};
%Text
%\node (face.275)[yshift=-3in] (text){\Huge \textbf{\LaTeX}};

% added by alan
%\draw (8,0) arc [start angle=0, end angle=270,
%x radius=1cm, y radius=5mm] -- cycle;

\end{pgfonlayer}

\begin{pgfonlayer} {background}
%Ears
%\draw [line width=4mm] (4.4,1.3) arc (-80:315:1);
%\draw [line width=4mm] (-4.3,1.3) arc (-80:315:1);
%hands
%\draw [line width=4mm] (3.05,-7.8) arc (-70:80:2.3);
%\draw [line width=4mm] (-3.05,-7.8) arc (250:90:2.3);

\end{pgfonlayer}
\end{tikzpicture}

\end{document}
